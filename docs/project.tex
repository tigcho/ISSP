\documentclass[11pt,a4paper]{article}

\usepackage[utf8]{inputenc}
\usepackage{graphicx}
\usepackage[colorlinks=true, 
            linkcolor=blue, 
            urlcolor=blue, 
            citecolor=blue, 
            anchorcolor=blue]{hyperref}
\usepackage[margin=2.5cm]{geometry}
\usepackage{enumitem}
\usepackage{titlesec}
\usepackage{xcolor}

\definecolor{sectioncolor}{RGB}{0, 83, 138}

\titleformat{\section}
{\color{sectioncolor}\normalfont\Large\bfseries}
{\thesection}{1em}{}

\title{\textbf{AWS AD Lab Project Plan}}
\author{Selin Kabak}

\begin{document}
\maketitle

\section{Project Overview}
\subsection{Description}
This hands-on cybersecurity lab project focuses on implementing and securing Windows Server and Active Directory infrastructure in AWS using Infrastructure-as-Code principles. The project utilizes Terraform for infrastructure provisioning, Vagrant for development environments, and Ansible for configuration automation, providing a practical learning environment for cloud security concepts.

\subsection{Objectives}
This hands-on lab project will teach me how to set up and manage Active Directory in AWS cloud. I will learn how to build a complete test environment that mirrors real-world setups, but in a way that's safe for learning and experimenting.

I will start by learning how to use Terraform to create the basic AWS building blocks - like virtual networks (VPCs). Using Vagrant, I'll create a local test environment on my computer where I can practice configurations safely before trying them in AWS.

For the Windows Server part, I'll learn how to automatically set up and configure servers using Ansible. This includes installing Active Directory, creating user accounts, setting up security policies, and making sure everything is properly secured. I'll focus on understanding how Active Directory works in the cloud, including how to set up domain controllers and manage users and groups.

The project will teach me about common security problems in Active Directory and how to fix them. This includes setting up proper backups, monitoring for security issues, and knowing what to do if something goes wrong. I'll also learn how to properly test my setups and keep track of changes using version control.

\section{Steps to Take}

\subsection{Learning the Tools}
Before diving into the actual implementation, I need to understand the basic tools:

\begin{enumerate}[label=\arabic*.]
    \item Learn AWS fundamentals
        \begin{itemize}
            \item Get familiar with VPC, EC2, Security Groups through the AWS Console
        \end{itemize}
    
    \item Get comfortable with Terraform
        \begin{itemize}
            \item Practice creating and destroying simple resources
            \item Learn about state files and variable management
        \end{itemize}
    
    \item Set up Vagrant environment
        \begin{itemize}
            \item Create a simple Windows Server VM
            \item Learn basic Vagrant commands and configuration
        \end{itemize}
    
    \item Learn Ansible basics
        \begin{itemize}
            \item Practice with simple playbooks
            \item Learn Windows-specific Ansible modules
        \end{itemize}
\end{enumerate}

\subsection{Implementation Tracks}

\textbf{Track 1: Local Testing Environment}
\begin{itemize}
    \item Create Vagrant file for Windows Server VM
    \item Set up local Active Directory in VM
    \item Practice AD management tasks locally
    \item Test Ansible playbooks against local VM
\end{itemize}

\textbf{Track 2: AWS Infrastructure
}
\begin{itemize}
    \item Write basic Terraform config for VPC
    \item Add security groups and subnets
    \item Create Windows Server EC2 instance
    \item Test connectivity and access
\end{itemize}

\textbf{Track 3: Automation Development}
\begin{itemize}
    \item Write Ansible playbooks for Windows setup
    \item Create AD installation and configuration scripts
    \item Develop user and group management automation
    \item Test automation scripts locally first
\end{itemize}

\subsection{Integration Points}
Key moments where different tracks come together:

\begin{itemize}
    \item When local AD setup works, move configurations to AWS
    \item Test Ansible playbooks on AWS instances
    \item Combine Terraform and Ansible for full automation
\end{itemize}

\subsection{Learning Notes}
Things to document as I go:

\begin{itemize}
    \item Tool installation steps and issues faced
    \item Common AWS and AD configuration problems
    \item Successful configurations and why they work
\end{itemize}

\section{Resources \& References}
\begin{itemize}
    \item[TOOLS]
    \item \href{https://xapax.github.io/security/}{Attacking Active Directory}
    \item \href{https://github.com/mdecrevoisier/SIGMA-detection-rules}{SIGMA detection rules for SIEM}
    \item \href{https://github.com/inspec/inspec}{InSpec: Infrastructure Testing Framework}
    \item \href{https://github.com/ansible-collections/vmware.vmware}{Official VMware Automation Collection for Ansible}
    \item \href{https://spacelift.io/blog/ansible-best-practices}{Ansible Best Practices and Standards}
    \item \href{https://www.frostbyte.us/configure-an-ansible-testing-system-on-windows-part-1/}{Setting Up Ansible Test Environment on Windows}
    \item \href{https://github.com/guillermo-musumeci/terraform-aws-ad-mfa-radius-linotp}{AWS AD with MFA using Terraform}
    
    \item[EXAMPLES]
    \item \href{https://github.com/gr33nm0nk2802/DelegationLab?tab=readme-ov-file}{AD Delegation Lab}
    \item \href{https://github.com/alebov/AD-lab}{AD Lab}
    \item \href{https://github.com/jckhmr/adlab?tab=readme-ov-file}{AD Lab with Attack Scenarios}
    \item \href{https://github.com/C4l1b4n/Active-Directory-Ansible-Lab}{Automated AD Lab Deployment using Ansible}
    \item \href{https://github.com/blink-zero/ansible-ad-lab}{Multi-Platform AD Environment with Ansible}
    \item \href{https://github.com/ldelouw/active-directory-ansible-playbook}{AD Setup with Groups and Users}
    \item \href{https://github.com/Cinderblook/vSphere-WinServ-Deployment}{Automated Windows Server Deployment in vSphere}
    
    \item [DOCS]
    \item \href{https://www.terraform.io/docs/providers/aws/}{Terraform Documentation}
    \item \href{https://developer.hashicorp.com/vagrant/docs}{Vagrant Documentation}
    \item \href{https://docs.ansible.com/ansible/latest/os_guide/windows_usage.html}{Ansible Windows Integration}

\end{itemize}
\end{document}